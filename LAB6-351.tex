\documentclass[12pt,a4paper]{article}

\usepackage[utf8]{inputenc}
\usepackage[greek,english]{babel}
\usepackage{alphabeta} 

\usepackage[pdftex]{graphicx}
\usepackage[top=1in, bottom=1in, left=1in, right=1in]{geometry}

\linespread{1.06}
\setlength{\parskip}{8pt plus2pt minus2pt}

\widowpenalty 10000
\clubpenalty 10000

\newcommand{\eat}[1]{}
\newcommand{\HRule}{\rule{\linewidth}{0.5mm}}

\usepackage[official]{eurosym}
\usepackage{enumitem}
\setlist{nolistsep,noitemsep}
\usepackage[hidelinks]{hyperref}
\usepackage{cite}
\usepackage{lipsum}


\begin{document}

%===========================================================
\begin{titlepage}
\begin{center}

% Top 
\includegraphics[width=0.55\textwidth]{cut-logo-en}~\\[2cm]


% Title
\HRule \\[0.4cm]
{ \LARGE 
  \textbf{Project Report for ECE 351}\\[0.4cm]
  \emph{Lab 6 - Partial Fraction Expansion}\\[0.4cm]
}
\HRule \\[1.5cm]



% Author
{ \large
  Sara Gergen \\[0.1cm]
  2/28/2023\\[0.1cm]
  %#\texttt{user@cut.ac.cy}
}

\vfill

%\textsc{\Large Cyprus University of Technology}\\[0.4cm]\textsc{\large Department of Electrical Engineering,\\Computer Engineering \& Informatics}\\[0.4cm]


\end{center}
\end{titlepage}

%\begin{abstract}
%\lipsum[1-2]
%\addtocontents{toc}{\protect\thispagestyle{empty}}
%\end{abstract}

\newpage



%===========================================================
\tableofcontents
\addtocontents{toc}{\protect\thispagestyle{empty}}
\newpage
\setcounter{page}{1}

%===========================================================
%===========================================================
\section{Introduction}\label{sec:intro}


\paragraph{The purpose of this lab is to use the script signal.residue() function in order to perform partial fraction expansion.}


\section{Equations}\label{sec:lit-rev}
\paragraph{Equation 1: Given Second Order Equation}

$$y''(t) + 10y'(t) + 24y(t) = x''(t) + 6x'(t) + 12x(t)$$

\paragraph{Equation 2: Found Transfer function }
$$H(s) = -6/(s+6) + 2/(s+4)$$
$$y(t) = [-6e^-^6^t + 2e^-^4^t]u(t) $$

\paragraph{Equation 3: y(t) for a step input with partial fraction expansion and inverse Laplace transformers}

$$H(s)* 1/s = 1/(s+6)- 0.5/(s+4)+0.5/s$$



\section{Methodology}\label{sec:meth}

\subparagraph{\Large Pre-lab}

\paragraph{Before the lab, Laplace transforms, and partial fraction expansion was used to find the step response of a given second-order equation. }
\newpage



\subparagraph{\Large Part 1}

\paragraph{This section of the lab plots the step response calculated from the pre-lab portion and the function by using the sig.residue() function to perform partial fraction expansion.}

\subparagraph{\Large Part 2}

\paragraph{The second part is for using the sig.residue() function for partial fraction expansion (a function that would be otherwise difficult to analyze). The difficult given function is the following:}

$$y'''''(t) + 18y''''(t) + 218y'''(t) + 2036y''(t) + 9085y'(t) + 25250y(t) = 25250x(t)$$

\paragraph{For the first step of the second part, the given sig.residue() function will be used once again in order to find the partial fraction expansion results; the results will then be printed (they are shown in the results section of the lab.
\newline
After the previous step above is completed, the next portion is to plot the time-domain response from 0 to 4.5 seconds with the cosine method. This method can be found in the book, Signals and Systems for Electrical Engineers 1, on page 87.}

\paragraph{The final step of part 2 is to check the response by using the sig.step() function and plot the results.}


\section{Results}\label{sec:res}

\paragraph{The first graph is the plot generated from the hand calculations}

\paragraph{
\begin{figure}[htp]
    \centering
    \includegraphics[width=10cm]{sr_hand.png}
    \caption{Step Response from Hand-Calculations}
    \label{fig:sr_hand.png}
\end{figure}
}

\paragraph{The second graph is from the sig.step() function; as you can see, it matches the hand-calculated function.}
\paragraph{
\begin{figure}[htp]
    \centering
    \includegraphics[width=10cm]{sig_res.png}
    \caption{Step Response from Hand-Calculations}
    \label{fig:sig_res.png}
\end{figure}
}

\paragraph{From using the sig.residue function, the following are the partial expansion results:}
$$R =  [ 2. -6.], P =  [-4. -6.], K =  [1.]$$

\subparagraph{\Large Part 2 Data}

\paragraph{This is the plotted time-domain response by using the cosine method.}
\paragraph{
\begin{figure}[htp]
    \centering
    \includegraphics[width=10cm]{cosMeth.png}
    \caption{Response with Cosine Method.}
    \label{fig:cosMeth.png}
\end{figure}
}

\paragraph{This is to check the response from the graph above.}
\paragraph{
\begin{figure}[htp]
    \centering
    \includegraphics[width=10cm]{sig_step_2.png}
    \caption{Response with Cosine Method.}
    \label{fig:sig_step_2.png.png}
\end{figure}
}


\paragraph{The printed outcome of the partial fraction expansion:\newline}


$R2 =  [ 3.18836328-6.87132672e-18j -1.82697653+2.63411515e+00j
 -1.82697653-2.63411515e+00j  0.23279489+6.22927457e-01j
  0.23279489-6.22927457e-01j]$
\newline
$P2 =  [-0.3002522  +0.j         -8.11349739 +2.85253391j
 -8.11349739 -2.85253391j -0.73637652+10.69014575j
 -0.73637652-10.69014575j]$
$$K2 =  []$$

\section{Error Analysis}\label{sec:res}

\paragraph{There is an error in the graph of my function graphed using the cosine method. Likely, there is an error in the function itself; however, the graph is somewhat close in shape, but the magnitude was off.}

\section{Questions}\label{sec:res}
\subparagraph{\Large 1. For a non-complex pole-residue term, you can still use the cosine method, explains why this works.}

\paragraph{This works because there will be no negative bounds inside the cosine function; thus, the cosine method will still work.}

\subparagraph{\Large 2. Leave any feedback on the clarity of the expectations, instructions, and deliverables.}
\paragraph{The lab was fairly clear in the instructions, in fact, I thought this lab was a bit easier to follow than other labs.}

\section{Conclusion}\label{sec:res}

\paragraph{This lab demonstrated the built-in functions such as sig.step() and sig.residue(). From this introduction of sig.residue() function, this provided a much easier way to solve for complicated partial fractions.}


%\lipsum[7-8]\cite{knuthwebsite}

%===========================================================
%===========================================================

\bibliographystyle{ieeetr}
\bibliography{refs}


\end{document} 



