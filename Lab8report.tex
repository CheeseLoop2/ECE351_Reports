\documentclass[12pt,a4paper]{article}

\usepackage[utf8]{inputenc}
\usepackage[greek,english]{babel}
\usepackage{alphabeta} 

\usepackage[pdftex]{graphicx}
\usepackage[top=1in, bottom=1in, left=1in, right=1in]{geometry}

\linespread{1.06}
\setlength{\parskip}{8pt plus2pt minus2pt}

\widowpenalty 10000
\clubpenalty 10000

\newcommand{\eat}[1]{}
\newcommand{\HRule}{\rule{\linewidth}{0.5mm}}

\usepackage[official]{eurosym}
\usepackage{enumitem}
\setlist{nolistsep,noitemsep}
\usepackage[hidelinks]{hyperref}
\usepackage{cite}
\usepackage{lipsum}


\begin{document}

%===========================================================
\begin{titlepage}
\begin{center}

% Top 
\includegraphics[width=0.55\textwidth]{cut-logo-en}~\\[2cm]


% Title
\HRule \\[0.4cm]
{ \LARGE 
  \textbf{Project Report for ECE 351}\\[0.4cm]
  \emph{Lab 8 - Fourier Series Approximation of a Square Wave}\\[0.4cm]
}
\HRule \\[1. 5cm]



% Author
{ \large
  Sara Gergen \\[0.1cm]
  3/21/2023\\[0.1cm]
  %#\texttt{user@cut.ac.cy}
}

\vfill

%\textsc{\Large Cyprus University of Technology}\\[0.4cm]\textsc{\large Department of Electrical Engineering,\\Computer Engineering \& Informatics}\\[0.4cm]


\end{center}
\end{titlepage}

%\begin{abstract}
%\lipsum[1-2]
%\addtocontents{toc}{\protect\thispagestyle{empty}}
%\end{abstract}

\newpage



%===========================================================
\tableofcontents
\addtocontents{toc}{\protect\thispagestyle{empty}}
\newpage
\setcounter{page}{1}

%===========================================================
%===========================================================
\section{Introduction}\label{sec:intro}

\paragraph{This lab aims to use the Fourier Series to approximate periodic time-domain signals.}

\paragraph{
\begin{figure}[htp]
    \centering
    \includegraphics[width=10cm]{wave.png}
    \caption{Given Square Wave Function}
    \label{fig:wave.png}
\end{figure}
}



\section{Equations}\label{sec:lit-rev}

\paragraph{The three equations given are:}

$$Equation 1: x(t) = \frac{1}{2}a_0 + \sum_{n = 1}^{\infty}a_kcos(kw_0t) + b_ksin(kw_0t)$$

$$Equation\space2: a_k = \frac{2}{T}\int_0^Tx(t)cos(kw_0t)dt$$

$$Equation\space3: b_k = \frac{2}{T}\int_0^Tx(t)sin(kw_0t)dt$$

$$Equation\space4: w_0 = \frac{2\pi}{T}$$

\paragraph{From the prelab:}
$$Equation_5: b_k = \frac{2}{k\pi}[1-cos(k\pi)]$$

\section{Methodology}\label{sec:meth}

\subparagraph{\Large Part 1.1}

\paragraph{After considering the following square wave function, assume that it can be represented by the Fourier series given in equation 1 where equation 2 and 3 apply, along with equation 4. First input the expressions a_k and b_k and use Python to solve for the first two a terms and the first three b terms and display a numerical values for each of the found terms, which are shown in the results section of the report.}

\subparagraph{\Large Part 1.2}
\paragraph{Using the square wave given, plot the Fourier series for an array of numbers given:}
$$N = {1, 3, 15, 50, 150, 1500}$$
\paragraph{Using equation 1 in the equation section, the period is to be set to T = 8 and the approximation plotted for a time interval between 0 to 20 seconds.}

\section{Results}\label{sec:res}

\newpage

\paragraph{The following are the printed results of part 1 task 1: \newline}
\paragraph{First terms:}
$$k =  1$$
$$a_1 =  0$$
$$b_1 =  1.2732395447351628$$
\paragraph{Second Terms:}
$$k =  2$$
$$a_2 =  0$$
$$b_2 =  0.0$$
\paragraph{Third Terms:}
$$k =  3$$
$$a_3 =  0$$
$$b_3 =  0.4244131815783876$$


\paragraph{The next part of the lab requires for the results to be plotted with the corresponding n value labeled: }

\paragraph{
\begin{figure}[htp]
    \centering
    \includegraphics[width=10cm]{n1.png}
    \caption{n1}
    \label{fig:n1.png}
\end{figure}
}
\paragraph{
\begin{figure}[htp]
    \centering
    \includegraphics[width=10cm]{n3.png}
    \caption{n3}
    \label{fig:n3.png}
\end{figure}
}
\paragraph{
\begin{figure}[htp]
    \centering
    \includegraphics[width=10cm]{n15.png}
    \caption{n15}
    \label{fig:n15.png}
\end{figure}
}
\paragraph{
\begin{figure}[htp]
    \centering
    \includegraphics[width=10cm]{n50.png}
    \caption{n50}
    \label{fig:n50.png}
\end{figure}
}
\paragraph{
\begin{figure}[htp]
    \centering
    \includegraphics[width=10cm]{n150.png}
    \caption{n150}
    \label{fig:n150.png}
\end{figure}
}
\paragraph{
\begin{figure}[htp]
    \centering
    \includegraphics[width=10cm]{n1500.png}
    \caption{n1500}
    \label{fig:n1500.png}
\end{figure}
}
\newpage

\section{Error Analysis}\label{sec:res}

\paragraph{The errors that happened were WOW?}

\section{Questions}\label{sec:res}
\subparagraph{\large 1.Is x(t) an even or an odd function? Explain why.}

\paragraph{The given function resembles that of a cosine function, which means that the function is even. Additionally, it is an even because of the definition, which is x(-t) = -x(t) and this is clearly true when looking at the graph.}


\subparagraph{\large 2. Based on your results from Task 1, what do you expect the values of a2, a3, . . . , an to be? Why?}

\paragraph{The results gotten were expected. From the task 1, it can be see that all values of a_{k} will be equal to zero.}

\subparagraph{\large 3. How does the approximation of the square wave change as the value of N increases? In what way does the Fourier series struggle to approximate the square wave? }

\paragraph{The approximation of the square wave starts off as a regular sine wave and then once n is increased the shape becomes closer to the true square wave shape.}

\subparagraph{\large 4. What is occurring mathematically in the Fourier series summation as the value of N increases?}

\paragraph{Mathematically, as the value of n increases, this increases the number of Fourier series produced, which brings the graph to a more accurate outline of the graph, which is increasing the accuracy.}

\subparagraph{\large 5. Leave any feedback on the clarity of lab tasks, expectations, and deliverables.}

\paragraph{this lab was rather straightforward and I did not encounter many issues.}

\section{Conclusion}\label{sec:res}

\paragraph{This lab clearly proved to show the utilization of the Fourier series. Using multiple iterations of the Fourier series allows for a graph to be made with relative accuracy. This is case, a square wave was create with relative accuracy using the Fourier series, when n is a higher value.}


%\lipsum[7-8]\cite{knuthwebsite}

%=======
%====================================================
%===========================================================

\bibliographystyle{ieeetr}
\bibliography{refs}


\end{document} 



