\documentclass[12pt,a4paper]{article}

\usepackage[utf8]{inputenc}
\usepackage[greek,english]{babel}
\usepackage{alphabeta} 

\usepackage[pdftex]{graphicx}
\usepackage[top=1in, bottom=1in, left=1in, right=1in]{geometry}

\linespread{1.06}
\setlength{\parskip}{8pt plus2pt minus2pt}

\widowpenalty 10000
\clubpenalty 10000

\newcommand{\eat}[1]{}
\newcommand{\HRule}{\rule{\linewidth}{0.5mm}}

\usepackage[official]{eurosym}
\usepackage{enumitem}
\setlist{nolistsep,noitemsep}
\usepackage[hidelinks]{hyperref}
\usepackage{cite}
\usepackage{lipsum}


\begin{document}

%===========================================================
\begin{titlepage}
\begin{center}

% Top 
\includegraphics[width=0.55\textwidth]{cut-logo-en}~\\[2cm]


% Title
\HRule \\[0.4cm]
{ \LARGE 
  \textbf{Project Report for ECE 351}\\[0.4cm]
  \emph{Introduction to Python 3.x and LATEX, Lab1}\\[0.4cm]
}
\HRule \\[1.5cm]



% Author
{ \large
  Sara Gergen \\[0.1cm]
  1/24/2023\\[0.1cm]
  %#\texttt{user@cut.ac.cy}
}

\vfill

%\textsc{\Large Cyprus University of Technology}\\[0.4cm]\textsc{\large Department of Electrical Engineering,\\Computer Engineering \& Informatics}\\[0.4cm]


 
\end{center}
\end{titlepage}

%\begin{abstract}
%\lipsum[1-2]
%\addtocontents{toc}{\protect\thispagestyle{empty}}
%\end{abstract}

\newpage
%%%%%%%%%%%%%%%%%%%%%%%%%%%%%%%%%%%%%%%%%%%%%%%%%%%%%%%%%%%%%%%%
%                                                              %
% Sara Gergen                                                 %
% ECE 351 Section 53                                           %
% Lab 01                                                       %
% Due: 1/24/23                                                  %
%                                                              %
%%%%%%%%%%%%%%%%%%%%%%%%%%%%%%%%%%%%%%%%%%%%%%%%%%%%%%%%%%%%%%%%

{\LARGE\header{Introduction}}


\textbf{In this lab, we will learn about Spyder and summarize the various sections.}


\section{Part one}

\textbf{I read over the different shortcuts for Spyder along with looking at the interactive tutorials given} 

\section{Part 2}

\textbf{In this section, I was able to type out the code and execute it without error. The console displayed the anticipated output. I learned how variables needed to be separated by a comma when within the print execution. I also learned that squaring a variable is not the same as C++, instead, you can just put "**" next to the exponent. I also went over arrays; however,in python, they are called lists. I also went over row vectors and column vectors.
This same part also went over comments 
This section also went over defining zero matrices and one matrices. It showed me how to also plot a graph in python.
It also went over complete numbers, which can be printed/generated in three different ways. The lab also went over adding an imaginary number in order to fix a square root error with a negative inside of the function.
The lab also went over the various packages that will be used throughout this semester.}

\section{Part 3}

\textbf{In this section, the purpose was to become familiar with pep8 coding practices. I am to use 4 spaces per indentation with no tabs. Tabs are used for if and else statements. Another practice I learned was to use docstrings to define functions/programs. Also, I learned that I can wrapped text, and that it shouldn't go past 79 characters in practice.
I learned more about how you should comment and the spaces you should use around operators and after commas.
}
\section{Part 4}

\textbf{I read through the LATEX cheat sheets and learned how to actually pronounce LATEX correctly. }

\section{Questions:}

\textbf{1. Which Course are you most excited for in your degree? Which course have you enjoyed the most so far?}

{\indent \textbf{I am most excited for 427 which is a class where I get to do a group design of a PCB. I have enjoyed energy systems 1 the most out of any class because I really enjoyed the hands-on labs with all the equipment provided in the lab, and I learned a lot in that class.} }



\textbf{2. Leave any feedback on the clarity of the expectations, instructions, and deliverables.}


\textbf{I'm not too sure if I had many criticisms because this lab seemed pretty clear.}

\indent \textbf{}

\bibliographystyle{ieeetr}
\bibliography{refs}


\end{document} 



