\documentclass[12pt,a4paper]{article}

\usepackage[utf8]{inputenc}
\usepackage[greek,english]{babel}
\usepackage{alphabeta} 

\usepackage[pdftex]{graphicx}
\usepackage[top=1in, bottom=1in, left=1in, right=1in]{geometry}

\linespread{1.06}
\setlength{\parskip}{8pt plus2pt minus2pt}

\widowpenalty 10000
\clubpenalty 10000

\newcommand{\eat}[1]{}
\newcommand{\HRule}{\rule{\linewidth}{0.5mm}}

\usepackage[official]{eurosym}
\usepackage{enumitem}
\setlist{nolistsep,noitemsep}
\usepackage[hidelinks]{hyperref}
\usepackage{cite}
\usepackage{lipsum}


\begin{document}

%===========================================================
\begin{titlepage}
\begin{center}

% Top 
\includegraphics[width=0.55\textwidth]{cut-logo-en}~\\[2cm]


% Title
\HRule \\[0.4cm]
{ \LARGE 
  \textbf{Project Report for ECE 351}\\[0.4cm]
  \emph{Lab 11 - Z Transform Operations}\\[0.4cm]
}
\HRule \\[1.5cm]



% Author
{ \large
  Sara Gergen \\[0.1cm]
  4/11/2023\\[0.1cm]
  %#\texttt{user@cut.ac.cy}
}

\vfill

%\textsc{\Large Cyprus University of Technology}\\[0.4cm]\textsc{\large Department of Electrical Engineering,\\Computer Engineering \& Informatics}\\[0.4cm]


\end{center}
\end{titlepage}

%\begin{abstract}
%\lipsum[1-2]
%\addtocontents{toc}{\protect\thispagestyle{empty}}
%\end{abstract}

\newpage



%===========================================================
\tableofcontents
\addtocontents{toc}{\protect\thispagestyle{empty}}
\newpage
\setcounter{page}{1}

%===========================================================
%===========================================================
\section{Introduction}\label{sec:intro}

\paragraph{the purpose of this lab is to analyze a discrete system with Python and it's built-in functions.}

\section{Equations}\label{sec:lit-rev}


\paragraph{The following is the casual function given: }
$y(k) = 2x(k) - 40x(k - 1) + 10y(k - 1) - 16y(k-2)$

\paragraph{The transfer function found from the given casual function:}
$$H = \frac{2z^{2}-40z}{z^2 - 10z + 16}$$

\paragraph{h[k] found with partial fraction expansion:}
$$h[k] = -4(8)^k+ 6(2)^k$$


\section{Methodology}\label{sec:meth}

\subparagraph{\Large Task 1:}

\paragraph{The following is finding the transfer function H(z), the derivation of task 1:}
$$y[k] = 2x[k] - 40x[k-1] + 10y[k-1] - 16[k-2]$$
$$Y = 2X - 40X z^{-1} + 10Yz^{-1} - 16Yz^{-2}$$
$$Y - 10Yz^{-1} + 16Yz^{-2} = 2X - 40X z^{-1}$$
$$\frac{Y}{X} = \frac{2-40z^{-1}}{1 - 10z^{-1} + 16z^{-2}}$$
\paragraph{Normalizing:}

$$H = \frac{2z^{2}-40z}{z^2 - 10z + 16}$$

\subparagraph{\large Task 2}
\paragraph{Finding h[k] with partial fraction expansion.}

$$H = \frac{2z-40}{z^2 - 10z + 16}$$

$$\frac{H}{z} = \frac{2z-40}{z^2 - 10z + 16}$$

$$\frac{H}{z} = \frac{2z-40}{(z - 8)(z - 2)}$$
$$\frac{H}{z} = \frac{A}{z-8} + \frac{B}{z-2}$$
$$A = \frac{2z-40}{z-2} Evaluated at {z=8}$$
$$B = \frac{2z-40}{z-8} Evaluated at z = 2$$
$$A = -4$$
$$B = 6$$
$$h[k] = -4(8)^k+ 6(2)^k$$

\subparagraph{\large Task 3}

\paragraph{The function sig.residuez() was used in order to confirm the results found in the previous task; it was verified to be correct, and the results are shown in the \textbf{Results} section.}

\subparagraph{\large Task 4}

\paragraph{In this task, the provided zplane() function, which was created by Christopher Felton, will be used to obtain the pole-zero plot for H(z).}

\subparagraph{\large Task 5}
\paragraph{In this final task, the function sig.freqz() will be used in order to plot the magnitude and phase responses of H(z).}

\section{Results}\label{sec:res}

\paragraph{The following were generated by the sig.residuez() function:}

\paragraph{Residues Corresponding to poles:
        $$[ 6. -4.]$$
        Poles:
        $$[2. 8.]$$ 
        Coefficients: No coefficients were generated}
    

\paragraph{The plot from task 4, in which the plane() function was used in order to create the pole-zero plot for H(z)}
\paragraph{
\begin{figure}[htp]
    \centering
    \includegraphics[width=10cm]{task 4.png}
    \caption{Pole-Zero Plot}
    \label{fig:task 4.png}
\end{figure}
}

\paragraph{The bode plots from task 5:}
\paragraph{
\begin{figure}[htp]
    \centering
    \includegraphics[width=10cm]{magn.png}
    \caption{Plotted Magnitude}
    \label{fig:magn.png}
\end{figure}
}
\paragraph{
\begin{figure}[htp]
    \centering
    \includegraphics[width=10cm]{phase.png}
    \caption{Plotted Phase}
    \label{fig:phase.png}
\end{figure}
}

\newpage
\section{Error Analysis}\label{sec:res}

\paragraph{There were minimal errors in this lab because the lab instructions were so straightforward. The only issues were making sure the magnitude and phase plots had the correct x-axis.}

\section{Questions}\label{sec:res}
\subparagraph{\large 1. Looking at the plot generated in Task 4, is H(z) stable? Explain why or why not.}

\paragraph{The plot generated in task 4 shows that H(z) is unstable since the poles and zeros are not even encompassed by the unit circle.}


\subparagraph{\large 2. Leave any feedback on the clarity of lab tasks, expectations, and deliverables.}

\paragraph{This lab was rather straightforward, and the instructions were very clear.}

\section{Conclusion}\label{sec:res}

\paragraph{To conclude, the provided sig.residuez() function is a valuable tool in order to verify the partial fraction expansion answer along with detecting any potential algebraic errors. Although the process of partial fraction expansion is rather straightforward, having this function as a reference saves time and allows for avoiding mistakes. \newline
Moreover, the generated Pole-Zero plot is beneficial in assessing the stability of the system. It allowed for quickly identifying the pole and zeros visually and their impact on the system's response. \newline
The sig.freq() function provided a straightforward means of generating bode plots. Overall, these tools are useful for system analysis and design.}



%\lipsum[7-8]\cite{knuthwebsite}

%=======
%====================================================
%===========================================================

\bibliographystyle{ieeetr}
\bibliography{refs}


\end{document} 



