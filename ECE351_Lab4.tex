\documentclass[12pt,a4paper]{article}

\usepackage[utf8]{inputenc}
\usepackage[greek,english]{babel}
\usepackage{alphabeta} 

\usepackage[pdftex]{graphicx}
\usepackage[top=1in, bottom=1in, left=1in, right=1in]{geometry}

\linespread{1.06}
\setlength{\parskip}{8pt plus2pt minus2pt}

\widowpenalty 10000
\clubpenalty 10000

\newcommand{\eat}[1]{}
\newcommand{\HRule}{\rule{\linewidth}{0.5mm}}

\usepackage[official]{eurosym}
\usepackage{enumitem}
\setlist{nolistsep,noitemsep}
\usepackage[hidelinks]{hyperref}
\usepackage{cite}
\usepackage{lipsum}


\begin{document}

%===========================================================
\begin{titlepage}
\begin{center}

% Top 
\includegraphics[width=0.55\textwidth]{cut-logo-en}~\\[2cm]


% Title
\HRule \\[0.4cm]
{ \LARGE 
  \textbf{Project Report for ECE 351}\\[0.4cm]
  \emph{Lab 4 - System Step Response Using Convolution}\\[0.4cm]
}
\HRule \\[1.5cm]



% Author
{ \large
  Sara Gergen \\[0.1cm]
  2/14/2023\\[0.1cm]
  %#\texttt{user@cut.ac.cy}
}

\vfill

%\textsc{\Large Cyprus University of Technology}\\[0.4cm]\textsc{\large Department of Electrical Engineering,\\Computer Engineering \& Informatics}\\[0.4cm]


 
\end{center}
\end{titlepage}

%\begin{abstract}
%\lipsum[1-2]
%\addtocontents{toc}{\protect\thispagestyle{empty}}
%\end{abstract}

\newpage



%===========================================================
\tableofcontents
\addtocontents{toc}{\protect\thispagestyle{empty}}
\newpage
\setcounter{page}{1}

%===========================================================
%===========================================================
\section{Introduction}\label{sec:intro}

\paragraph{In this lab, the goal was further to familiarize ourselves with convolutions by creating convolution user-defined functions and convolving functions by hand and comparing the two, along with coding with Python.}


\section{Equations}\label{sec:lit-rev}

\paragraph{The following equations represented the signals created as user-defined functions}

$$h_1(t) = (u(t)-u(t-3))e^{-2t}$$
$$h_2(t) = u(t-2) - u(t-6)$$ 
$$h_3(t) = cos(wt)u(t)$$

\paragraph{Hand derived equations:}

$$f_1 = -1/2(1-e^{2t})u(t) + 1/2(1 - e^{2(t-3)})u(t-3)$$
$$f_2 = (t-2)u(t-2) - (t-6)u(t-6)$$
$$f_3 = (1/w)*(cos(wt)sin(wt) - sin(wt)(cos(tw)-1))u(t)$$
$$where f = 0.25 Hz$$

\section{Methodology}\label{sec:meth}

\subparagraph{\large Part 1}

\paragraph{The first step of this part is to use the step function that was coded in the previous lab in order to write out the equations for the first task. After that is completed, the user-defined functions, given in the equation section, are to be written and graphed.}

\subparagraph{\large Part 2}

\paragraph{Using the user-defined function for a convolution made in the last lab, find the step response of the three transfer functions given. Each of these step responses are to be plotted. Next, a hand calculation is to be done on each of the given functions in order to find the step response. Graph each of these found functions and compare them to the step responses which were found by using the user-defined function.}





\section{Results}\label{sec:res}

\paragraph{The first task of plotting the user-defined functions, which are listed in the equations section. The functions h1, h2, and h3 can be seen in figures 1-3. }

\paragraph{
\begin{figure}[htp]
    \centering
    \includegraphics[width=10cm]{h1.png}
    \caption{Function 1}
    \label{fig:h1.png}
\end{figure}
}
\paragraph{
\begin{figure}[htp]
    \centering
    \includegraphics[width=10cm]{h2.png}
    \caption{Function 2}
    \label{fig:h2.png}
\end{figure}
}
\paragraph{
\begin{figure}[htp]
    \centering
    \includegraphics[width=10cm]{h3.png}
    \caption{Function 3}
    \label{fig:h3.png}
\end{figure}
}


\paragraph{These are the graphs of the found calculations using the user-defined convolution function:} 


\paragraph{
\begin{figure}[htp]
    \centering
    \includegraphics[width=10cm]{con1.png}
    \caption{Step Response of Function 1}
    \label{fig:con1.png}
\end{figure}
}

\paragraph{
\begin{figure}[htp]
    \centering
    \includegraphics[width=10cm]{con2.png}
    \caption{Step Response of Function 2}
    \label{fig:con2.png}
\end{figure}
}

\paragraph{
\begin{figure}[htp]
    \centering
    \includegraphics[width=10cm]{con3.png}
    \caption{Step Response of Function 3}
    \label{fig:con3.png}
\end{figure}
}


\newpage 

\paragraph{The following figures are the hand derived functions graphed with Python:}
\paragraph{
\begin{figure}[htp]
    \centering
    \includegraphics[width=10cm]{hcon1.png}
    \caption{Hand Calculated Step Response, Function 1}
    \label{fig:h1.png}
\end{figure}
}


\paragraph{
\begin{figure}[htp]
    \centering
    \includegraphics[width=10cm]{hd2.png}
    \caption{Hand Calculated Step Response, Function 2}
    \label{fig:hd2.png}
\end{figure}
}


\paragraph{
\begin{figure}[htp]
    \centering
    \includegraphics[width=10cm]{hcon3.png}
    \caption{Hand Calculated Step Response, Function 3}
    \label{fig:3.png}
\end{figure}
}
\newpage

\section{Error Analysis}\label{sec:res}

\paragraph{In the case of my first hand derived function, shown in figure 7, I could not quite get it to match the function that was generated by my user-defined convolution. This could be either from me miscalculating it when doing it by hand, or mis-typing it as a function to plot, or my convolution function is off. Given that the two other convolutions worked out, I assume that I type it wrong as a function given, even though I typed it in a several different times and a several different ways.}

\section{Questions}\label{sec:res}

\subparagraph{\large Leave any feedback on the clarity of lab tasks, expectations, and deliverables.}

\paragraph{There is not much to criticize anymore. As I am becoming more familiar with the format of the lab instructions, they are getting much easier to read.}

\section{Conclusion}\label{sec:res}
%\lipsum[7-8]\cite{knuthwebsite}

\paragraph{Overall, this lab was helpful in solidifying my understanding of a convolution. I thought it was good to strengthen the fact that my calculations for step responses of the given functions, for the most part, matched my calculations done with my convolution function. }


%===========================================================
%===========================================================

\bibliographystyle{ieeetr}
\bibliography{refs}


\end{document} 



